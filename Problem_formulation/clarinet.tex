\section{CLARINET}

\subsection{Given claims}

The paper's introduction gives the following claims which need to be investigated:

\begin{enumerate}
	\item Formulating an optimal solution for a multi-query, network-aware, joint query planning/placement/scheduling is computationally intractable. \\
	$\checkmark$ Cites two sources that even for a single query with the given scheduling assumptions is NP-hard.
	\item CLARINET does joint multi-query planning, task placement, and task scheduling.
	It picks the best WAN-aware QEP/task placement/take scheduling per query and provides ``hints" to the query execution layer.\\
	$\checkmark$ ``hints" are location and start times of each task.
	\item They show how to (heuristically) compute the WAN-optimal QEP for a single query which includes task placement and task scheduling.
	The solution relies on reserving WAN links.
	This is an effective heuristic for the best single-query QEP, that decouples placement and scheduling.
	\item They allow for cross-query (heuristic) optimization of $n$ queries.
	They order the queries by optimal QEP expected completion time.
	Then they choose the $i$th query's QEP considering the WAN impact of the $i-1$ preceding queries.
	\item They heuristically compact the schedules tightly in time by considering the above order and groups of $k\leq n$ queries.
	\item They extend the above heuristic to accommodate (i) fair treatment of queries, (ii) minimize WAN bandwidth costs, and (iii) online query arrivals.
	\item Combats resource fragmentation and optimizes average query completion times.
\end{enumerate}

\subsection{Model}
There are $n$ queries and each query $j$ has a set QEP-Set $QS_j$ from which one QEP must be chosen.
Also chosen are the task locations and task start times.

Scheduling assumptions:
\begin{enumerate}
	\item Network transfers do not overlap.
	In fact, they have a theorem that proves that any optimal schedule has an equivalent non-overlapped schedule.
	\item Obtain non-interruptible transfer schedules.
\end{enumerate}