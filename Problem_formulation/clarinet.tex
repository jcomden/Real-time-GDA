\section{CLARINET}

\subsection{Given claims}

The paper's introduction gives the following claims which need to be investigated:

\begin{enumerate}
	\item Formulating an optimal solution for a multi-query, network-aware, joint query planning/placement/scheduling is computationally intractable.
	\item CLARINET does joint multi-query planning, task placement, and task scheduling.
	It picks the best WAN-aware QEP/task placement/take scheduling per query and provides ``hints" to the query execution layer.
	\item They show how to compute the WAN-optimal QEP for a single query which includes task placement and task scheduling.
	The solution relies on reserving WAN links.
	\item They allow for cross-query optimization of $n$ queries.
	They order the queries by optimal QEP expected completion time.
	Then they choose the $i$th query's QEP considering the WAN impact of the $i-1$ preceding queries.
	\item They heuristically compact the schedules tightly in time by considering the above order and groups of $k\leq n$ queries.
	\item They extend the above heuristic to accommodate (i) fair treatment of queries, (ii) minimize WAN bandwidth costs, and (iii) online query arrivals.
	
\end{enumerate}