\section{Single DAG Query}

\subsection{Model}

\subsubsection*{Wide Area Network}

Let $\mathcal{S}$ be the set of sites that hold data and run tasks, and $\mathcal{L}$ be the set of directed edges that represent inter-site links.
For each inter-site WAN link $(i,j)\in\mathcal{L}$, let $B_{ij}$ be the bandwidth from site $i$ to site $j$.
We assume that the bandwidths are stable within the time frame of doing real-time data analytics.

\subsubsection*{DAG of tasks}

We define a stage in the DAG as a group of tasks that have the same input data dependencies.
Let $\mathcal{T}$ be the set of stages and $\mathcal{D}$ be the directed set of edges that represent stage dependencies for the DAG, respectively.
Each stage dependency $(k,l)\in\mathcal{D}$ also has a corresponding data amount $D^{kl}$ that must be transfered from stage $k$ to stage $l$.

Some of the stages do not have any incoming edges and so represent the raw input data.
Let $\mathcal{R}:=\left\{(k,i)\right\}$ be a set of stage and location pairs for the raw input data in which stage $k\in\mathcal{T}$ represents the input data and site $i\in\mathcal{S}$ is the location of this input data.

The DAG also has a start time $T_0$ and a finish time $T_f$ for which the schedule of data transfers must respect.

\subsubsection*{Stage assignment}

We first model each stage as a non-distributable group of tasks that must be assigned to one site.
This constraint will be relaxed in the problem formulation section.
The decision to place stage $k$ at site $i$ is represented by the binary decision variable $x_i^k$.
This means that for any directed edge $(k,l)\in\mathcal{D}$, if $x_i^k=1$ and $x_j^l=1$, then $D^{kl}$ of data will be transfered across link $(i,j)\in\mathcal{L}$.
We can represent this data transfer amount by $x_i^kD^{kl}x_j^l$ and the duration of data transfer by $x_i^k(D^{kl}/B_{ij})x_j^l$.
Note that stages which represent raw data locations (no incoming edges) have a decision variable which is preset according to its location.

\subsubsection*{Stage scheduling}

We assume that data transfers on a particular link are non-overlapping and non-interruptible.
Let $s_{ij}^{kl}$ be the scheduled start time for data transfer $(k,l)\in\mathcal{D}$ on link $(i,j)\in\mathcal{L}$.
We also assume that a stage must have completely received all of its input data before it can process and start transferring data to another stage.
For this reason we have the following scheduling constraint to respect DAG dependencies:
\begin{align}
	 \max_{k:(k,l)\in\mathcal{D}}\left\{\max_{(i,j)\in\mathcal{L}}\left\{s_{ij}^{kl}+x_{i}^k\dfrac{D^{kl}}{B_{ij}}x_{j}^l\right\}\right\} \leq \min_{(i,j)\in\mathcal{L}}\left\{s_{ij}^{lm}\right\} \nonumber \\
	 \forall(l,m)\in\mathcal{D} \label{eq:DAG_stage_start}
\end{align}
Also, each data transfer must respect the start and finish time $[T_0,T_f]$ of the DAG:
\begin{subequations}\label{eq:DAG_deadline}
	\begin{align}
		\min_{(k,l)\in\mathcal{D}}\left\{\min_{(i,j)\in\mathcal{L}}\left\{s_{ij}^{kl}\right\}\right\} & \geq T_0 \\
		\max_{(k,l)\in\mathcal{D}}\left\{\max_{(i,j)\in\mathcal{L}}\left\{s_{ij}^{kl}+x_{i}^k\dfrac{D^{kl}}{B_{ij}}x_{j}^l\right\}\right\} & \leq T_f
	\end{align}
\end{subequations}

There are data transfers that don't have an ordering because the DAG does not contain a path that connects them.
However, each link must have a specified ordering to schedule the data transfers.
Let $z_{ij}^{kl|mn}$ be the binary decision on link $(i,j)\in\mathcal{L}$ that represents data transfer $(k,l)\in\mathcal{D}$ being scheduled to start and complete before data transfer $(m,n)\in\mathcal{D}$ starts.
The following constraint defines this relation mathematically:
\begin{subequations}\label{eq:DAG_data_trans_start}
	\begin{align}
		s_{ij}^{kl}+x_{i}^k\dfrac{D^{kl}}{B_{ij}}x_{j}^l \leq s_{ij}^{mn} + M(1-z_{ij}^{kl|mn}) \quad \nonumber \\
		\quad\forall \{(k,l),(m,n)\} \in\mathcal{D}:(k,l)\neq(m,n), \forall(i,j)\in\mathcal{L}\\
		z_{ij}^{kl|mn} + z_{ij}^{mn|kl} = 1 \quad \nonumber\\
		\quad\forall \{(k,l),(m,n)\} \in\mathcal{D}:(k,l)\neq(m,n), \forall(i,j)\in\mathcal{L}
	\end{align}
\end{subequations}
where $M$ is a large constant (e.g. $M:=T_f-T_0$).



\subsection{Problem}

\subsubsection{General DAG - Minimize WAN usage}

Suppose that our objective is to minimize WAN usage by placing the stages from the DAG and scheduling the data transfers.  We get the following optimization problem:
\begin{subequations} \label{eq:opt_genDAG}
	\begin{align}
		\min_{\mathbf{x},\mathbf{z},\mathbf{s}} \quad & \sum_{(k,l)\in\mathcal{D}}\sum_{(i,j)\in\mathcal{L}}x_i^kD^{kl}x_j^l \label{eq:opt_genDAG_WAN} \\
		\text{s.t.} \quad & \sum_{i\in\mathcal{S}}x_i^k = 1 \quad \forall k\in\mathcal{T} \label{eq:opt_genDAG_place1} \\
		& x_i^k \in\{0,1\} \quad \forall i\in\mathcal{S},k\in\mathcal{T} \label{eq:opt_genDAG_placebin} \\
		& x_i^k = 1 \quad \forall (k,i)\in\mathcal{R} \label{eq:opt_genDAG_placeraw} \\
		& z_{ij}^{kl|mn} \in\{0,1\} \nonumber\\
		& \quad \forall \{(k,l),(m,n)\} \in\mathcal{D}:(k,l)\neq(m,n), \forall(i,j)\in\mathcal{L} \label{eq:opt_genDAG_ordersbin} \\
		& \eqref{eq:DAG_stage_start} \eqref{eq:DAG_deadline} \eqref{eq:DAG_data_trans_start} \label{eq:opt_genDAG_schedule}
	\end{align}
\end{subequations}
where \eqref{eq:opt_genDAG_WAN} is the total WAN usage, \eqref{eq:opt_genDAG_place1} \eqref{eq:opt_genDAG_placebin} place each task at exactly one site, \eqref{eq:opt_genDAG_placeraw} sets the locations of the raw data, \eqref{eq:opt_genDAG_ordersbin} \eqref{eq:opt_genDAG_schedule} are the scheduling constraints of the data transfers.

Notice that the objective function and first three constraints (\eqref{eq:opt_genDAG_WAN} \eqref{eq:opt_genDAG_place1} \eqref{eq:opt_genDAG_placebin} \eqref{eq:opt_genDAG_placeraw}) only depend on the stage placement decisions $\mathbf{x}$, while the following constraints (\eqref{eq:opt_genDAG_ordersbin} \eqref{eq:opt_genDAG_schedule}) depend on both the stage placement and scheduling $(\mathbf{z}$,$\mathbf{s})$ decisions.

\subsubsection{MapReduce Relaxation}
Suppose we relaxed the task placement decisions from being binary to nonnegative continuous, i.e. replacing \eqref{eq:opt_genDAG_placebin} with:
\begin{align}
	x_i^k \geq 0 \quad \forall i\in\mathcal{S},k\in\mathcal{T},
\end{align}
then the each stage fits the simplification used in a MapReduce model.
Also, if \eqref{eq:opt_genDAG_placeraw} can be set as fractions, then we can model a distributed data set based on its fraction at each location.

The MapReduce simplification works by allowing a reduce stage to be distributed across all sites where $r^l_j$ represents the fraction of reduce tasks for stage $l$ that are at site $j$.
This model assumes that the amount of input data needed at a site and the amount of output data produced is proportional to the fraction of reduce tasks placed there.
This means that if stage $k$ precedes stage $l$ in the DAG, stage $l$'s reduce tasks at site $j$ need a total of $D^{kl}r_j^l$ input data from all sites and the stage $k$'s reduce tasks at site $i$ produced the fraction $r_i^k$ of that total.
Therefore, $r_i^kD^{kl}r_j^l$ of data will be sent across link $(i,j)$ and take the time duration $r_i^k(D^{kl}/B_{ij}^{kl})r_j^l$ to transfer.

Replacing $x_i^k$ with $r_i^k$ and the MapReduce model in Problem \eqref{eq:opt_genDAG} is exactly the same as relaxing the binary task placement decision variables \eqref{eq:opt_genDAG_placebin}.

\subsection{Decoupling the problem}


The goal is to decouple the stage placement and scheduling problems so that there can be a metric from the optimal stage placement to help the scheduling problem.

\subsubsection{Stage placement to minimize WAN usage}
First we will simplify the WAN usage:
\begin{align}
	\sum_{(k,l)\in\mathcal{D}}\sum_{(i,j)\in\mathcal{L}}x_i^kD^{kl}x_j^l & = \sum_{(k,l)\in\mathcal{D}}D^{kl}\sum_{(i,j)\in\mathcal{L}}x_i^kx_j^l \nonumber\\
	& = \sum_{(k,l)\in\mathcal{D}}D^{kl}\sum_{i\in\mathcal{S}}\sum_{j\in\mathcal{S}:j\neq i}x_i^kx_j^l \nonumber\\
	& = \sum_{(k,l)\in\mathcal{D}}D^{kl}\left(\sum_{i\in\mathcal{S}}\sum_{j\in\mathcal{S}}x_i^kx_j^l-\sum_{i\in\mathcal{S}}x_i^kx_i^l\right) \nonumber\\
	& = \sum_{(k,l)\in\mathcal{D}}D^{kl}\left(1-\sum_{i\in\mathcal{S}}x_i^kx_i^l\right)
\end{align}
which means that the aggregate WAN usage measurement only depends on the product of stage placement fractions on the same sites for consecutive stages.

This gives the stage placement problem to be:
\begin{subequations}
	\begin{align}
		\min_{\mathbf{x}} \quad & \sum_{(k,l)\in\mathcal{D}}D^{kl}\left(1-\sum_{i\in\mathcal{S}}x_i^kx_i^l\right) \\
		\text{s.t.} \quad & \sum_{i\in\mathcal{S}}x_i^k = 1 \quad \forall k\in\mathcal{T} \\
		& x_i^k \geq 0 \quad \forall i\in\mathcal{S},k\in\mathcal{T} \\
		& x_i^k = 1 \quad \forall (k,i)\in\mathcal{R}
	\end{align}
\end{subequations}

\subsubsection{Data transfer scheduling}

Given the stage placements $\mathbf{x}$, we want to find the orderings $\mathbf{z}$ and start times $\mathbf{s}$ for the data transfers:
\begin{subequations}
	\begin{align}
		\min_{(k,l)\in\mathcal{D}}\left\{\min_{(i,j)\in\mathcal{L}}\left\{s_{ij}^{kl}\right\}\right\} & \geq T_0 \\
		\max_{(k,l)\in\mathcal{D}}\left\{\max_{(i,j)\in\mathcal{L}}\left\{s_{ij}^{kl}+x_{i}^k\dfrac{D^{kl}}{B_{ij}}x_{j}^l\right\}\right\} & \leq T_f
	\end{align}
	\begin{align}
		\max_{k:(k,l)\in\mathcal{D}}\left\{\max_{(i,j)\in\mathcal{L}}\left\{s_{ij}^{kl}+x_{i}^k\dfrac{D^{kl}}{B_{ij}}x_{j}^l\right\}\right\} \leq \min_{(i,j)\in\mathcal{L}}\left\{s_{ij}^{lm}\right\} \nonumber \\
		\forall(l,m)\in\mathcal{D}
	\end{align}
	\begin{align}
		s_{ij}^{kl}+x_{i}^k\dfrac{D^{kl}}{B_{ij}}x_{j}^l \leq s_{ij}^{mn} + M(1-z_{ij}^{kl|mn}) \quad \nonumber \\
		\quad\forall \{(k,l),(m,n)\} \in\mathcal{D}:(k,l)\neq(m,n), \forall(i,j)\in\mathcal{L}\\
		z_{ij}^{kl|mn} + z_{ij}^{mn|kl} = 1 \quad \nonumber\\
		\quad\forall \{(k,l),(m,n)\} \in\mathcal{D}:(k,l)\neq(m,n), \forall(i,j)\in\mathcal{L} \\
		z_{ij}^{kl|mn} \in\{0,1\} \nonumber\\
		\quad \forall \{(k,l),(m,n)\} \in\mathcal{D}:(k,l)\neq(m,n), \forall(i,j)\in\mathcal{L}
	\end{align}
\end{subequations}

Since scheduling is a complex problem, we would like to start with a greedy scheduling method (e.g. SRPT).
Our goal is to find $\mathbf{s}$ as a function of $\mathbf{z}$.