\section{Single DAG Query}

\subsection{Model}

\subsubsection*{Wide Area Network}

Let $\mathcal{S}$ be the set of sites that hold data and run tasks, and $\mathcal{L}$ be the set of directed edges that represent inter-site links.
For each inter-site WAN link $(i,j)\in\mathcal{L}$, let $B_{ij}$ be the bandwidth and $C_{ij}$ be the cost to transfer one unit of data from site $i$ to site $j$.
\begin{assumption}
	The bandwidths are stable within the time frame of real-time data analytics.
\end{assumption} 

For the analysis we make the following assumption:
\begin{assumption}
	Data transfers on a particular link are non-overlapping.
	\todo{Show that this assumption does not add suboptimality to our objective, i.e., any overlapping schedule can be transformed into a non-overlapping schedule without any loss to the optimal value.}
\end{assumption}

\subsubsection*{DAG of tasks}

We define a stage in the DAG as a group of tasks that have the same input data dependencies.
Let $\mathcal{T}$ be the set of stages and $\mathcal{D}$ be the directed set of edges that represent stage dependencies for the DAG, respectively.
Some of the stages do not have any incoming edges and so represent the raw input data.
Let $\mathcal{R}\subset\mathcal{T}$ be the set of stages for the raw input data.
Each stage dependency $(k,l)\in\mathcal{D}$ also has a corresponding amount of data $D^{kl}$ that must be transfered from stage $k$ to stage $l$.

\begin{assumption}
	A stage must have completely received all of its input data before it can process and start transferring data to another stage.
\end{assumption}

The DAG also has a start time $T_0$ and a finish time $T_f$ for which the schedule of data transfers must respect.

\subsection{Task Assignment Problem}

We model each stage as a distributable group of tasks that may be fractionally assigned to multiple sites.
The fraction of tasks from stage $k$ placed at site $i$ is represented by the nonnegative decision variable $x_i^k$ and proportionally determines the size of data transfers.
This means that for any directed edge $(k,l)\in\mathcal{D}$, then $x_i^kD^{kl}x_j^l$ of data will be transfered across link $(i,j)\in\mathcal{L}$.
Note that the set of stages $\mathcal{R}$ which represent the raw data have a decision variable which is preset according to its site-wise distribution.

We can represent the duration of a data transfer by $x_i^k(D^{kl}/B_{ij})x_j^l$.
Each data transfer is given an upper bound on its total link duration $d_{ij}^{kl}$.
This allows data transfers to be more generally scheduled and not need to be uninterruptible.
This is so that they can be scheduled by any general scheduling algorithm (e.g. SRPT).

The goal of the task assignment problem is to fractionally distribute the stages of tasks to minimize the WAN usage for a given set of link duration upper bounds:

\begin{subequations}
	\begin{align}
		\min_{\mathbf{x}} \quad & \sum_{(k,l)\in\mathcal{D}}\sum_{(i,j)\in\mathcal{L}}C_{ij}D^{kl}x_i^kx_j^l \\
		\text{s.t.}\quad & x_i^k(D^{kl}/B_{ij})x_j^l \leq d_{ij}^{kl} \quad \forall(i,j)\in\mathcal{L}, \forall(k,l)\in\mathcal{D} \\
		& \sum_{i\in\mathcal{S}}x_i^k = 1 \quad \forall k\in\mathcal{T} \\
		& x_i^k \geq 0 \quad \forall i\in\mathcal{S},k\in\mathcal{T}
	\end{align}
\end{subequations}

\subsection{Data Transfer Scheduling Problem}

Because of the dependencies specified by the DAG, each data transfer duration link allocation $d_{ij}^{kl}$ must be scheduled after the deadline of stage $k$ denoted $t^k$ and before the deadline of stage $l$ denoted $t^l$.
Therefore the goal is to find a scheduling policy $\pi$ and stage deadlines $\mathbf{t}$ such that the query start time $T_0$ and deadline $T_f$ are satisfied for the given set of durations $\mathbf{d}$ and stage dependencies $\mathcal{D}$.

\subsection{Jointly optimization}

Both the Task Assignment and Data Transfer Scheduling problems are connected through the data transfer duration link allocations $\mathbf{d}$.
Therefore the joint optimization problem becomes if we can find a feasible schedule that minimizes the WAN usage cost by adjusting $\mathbf{d}$.