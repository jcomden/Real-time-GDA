\section{Single DAG Query}

\subsection{Model}

\subsubsection{Wide Area Network}

Let $\mathcal{S}$ be the set of sites that hold data and run tasks, and $\mathcal{L}$ be the set of directed edges that represent inter-site links.
For each inter-site WAN link $(i,j)\in\mathcal{L}$, let $B_{ij}$ be the bandwidth and $C_{ij}$ be the cost to transfer one unit of data from site $i$ to site $j$.
\begin{assumption}
	The bandwidths are stable within the time frame of real-time data analytics.
\end{assumption} 

For the analysis we make the following assumption:
\begin{assumption}
	Data transfers on a particular link are non-overlapping.
	\todo{Show that this assumption does not add suboptimality to our objective, i.e., any overlapping schedule can be transformed into a non-overlapping schedule without any loss to the optimal value.}
\end{assumption}

%We don't need this assumption if we allow preemption:
%\begin{assumption}
%	Data transfers on a particular link are non-interruptible.
%	\todo{Show that this assumption does not add suboptimality to our objective, i.e., any schedule with interruptions can be transformed into a non-interruptible schedule without any loss to the optimal value.}
%\end{assumption}

\subsubsection{DAG of tasks}

We define a stage in the DAG as a group of tasks that have the same input data dependencies.
Let $\mathcal{T}$ be the set of stages and $\mathcal{D}$ be the directed set of edges that represent stage dependencies for the DAG, respectively.
Some of the stages do not have any incoming edges and so represent the locations of the raw input data.
Let $\mathcal{R}\subset\mathcal{T}$ be the set of stages for the raw input data.
The final stage in the DAG does not have any outgoing edges and so represent the final destination of the query's response denoted by $F\in\mathcal{T}$.
Each stage dependency $(k,l)\in\mathcal{D}$ also has a corresponding amount of data $D^{kl}$ that must be transfered from stage $k$ to stage $l$.

%We separate the DAG into a set $\mathcal{P}$ of disjoint paths $\mathcal{D}_m:m\in\mathcal{P}$ so that the union of paths make up the whole DAG.
%We also give an ordering (given by the index values) for the paths so that a path that precedes another path in the ordering gets precedence when there are link contentions.
%To make sure that the path ordering does not hold up the whole query execution, this ordering should lexicographically respect each path's stage ordering as given in the DAG starting from the end; meaning that a path with a final stage that comes later in all topological orderings of the DAG should be ordered before a path whose final stage comes sooner.
%If an ordering cannot be decided by the final stages, then compare the second to last stages and so on.

\begin{assumption}
	A stage must have completely received all of its input data before it can process and start transferring data to another stage.
\end{assumption}

The DAG also has a start time $T_0$ and a finish time $T_f$ for which the schedule of data transfers must respect.

\subsubsection{Stage assignment decisions}
We model each stage as a group of tasks that must be scheduled to run together at the same site.
The decision of stage $k$ to be placed at site $i$ is represented by the binary decision variable $x_i^k\in\{0,1\}$.
This means that for any directed edge $(k,l)\in\mathcal{D}$, then $D^{kl}x_i^kx_j^l$ of data will be transfered across link $(i,j)\in\mathcal{L}$.
Note that the set of stages $\mathcal{R}$ and $F$ which respectively represent the raw data and the DAG's final stage have decision variables which are preset according to its site-wise distribution and final stage location.

%%% Adding in decisions to block data transfers:
%Each stage dependency $(k,l)$ and pair of links $(i,j)\in\mathcal{L}$ have a binary decision variable $q_{ij}^{kl}\in\{0,1\}$ that determines whether link $(i,j)$ is available to be used by stage $(k,l)$ or not.

%%% Modeling the Data transfer scheduling decisions:
%\subsubsection{Scheduling decisions}
%For each link, a priority needs to be decided between all the data transfers on that link.
%Let $u_{ij}^{kl|mn}\in\{0,1\}$ be the binary decision to prioritize data transfer $(k,l)$ over $(m,n)$ on link $(i,j)$.
%Note that this decision variable is already set if there exists a directed path that can pass through both $(k,l)$ and $(m,n)$.

\subsection{Stage Assignment Problem}

\subsubsection{Problem Statement}

Given the available links for each dependency we have the following problem to minimize the cost of the WAN by deciding where to place each stage:

\begin{subequations}\label{eq:opt_stage}
	\begin{align}
		\min_{\mathbf{x}} \quad & \sum_{(k,l)\in\mathcal{D}}\sum_{(i,j)\in\mathcal{L}}C_{ij}D^{kl}x_i^kx_j^l \nonumber \\
		%\text{s.t.}\quad & x_i^kx_j^l \leq q_{ij}^{kl} \quad \forall (i,j)\in\mathcal{L}, \forall(k,l)\in\mathcal{D} \label{eq:opt_stage-duration} \\
		\text{s.t.}\quad & \sum_{i\in\mathcal{S}}x_i^k = 1 \quad \forall k\in\mathcal{T} \label{eq:opt_stage-sum1} \\
		& x_i^k \in \{0,1\} \quad \forall i\in\mathcal{S},k\in\mathcal{T} \label{eq:opt_stage-binary}
	\end{align}
\end{subequations}
Note that for any stage $k\in\mathcal{R}\cup F$, the placement decision is preset.

\subsection{Data Transfer Scheduling Problem}

The goal is to find a schedule $\pi$ given stage assignment $\mathbf{x}$ that satisfies the stage dependencies $\mathcal{D}$ from the DAG, the query's start time $T_0$ and deadline $T_f$.
Let $t_0^{kl}$ and $t_f^{kl}$ be the start and finish times to transfer data from stage $k$ to stage $l$.
Therefore, the schedule must satisfy:
\begin{subequations}
	\begin{align}
		\max_{k:(k,l)\in\mathcal{D}}t_f^{kl}(\pi(\mathbf{x})) & \leq \min_{m:(l,m)\in\mathcal{D}}t_0^{lm}(\pi(\mathbf{x})) \quad \forall l\in\mathcal{T} \\
		\min_{(k,l)\in\mathcal{D}}t_0^{kl}(\pi(\mathbf{x})) & \geq T_0 \\
		\max_{(k,l)\in\mathcal{D}}t_f^{kl}(\pi(\mathbf{x})) & \leq T_f
	\end{align}
\end{subequations}

The bandwidth constraints on each link must also be satisfied.
Let $d_{ij}^{kl}$ be the total amount of time (not necessarily contiguous) that link $(i,j)$ is devoted to transfer data from stage $k$ to stage $l$.
Therefore, the schedule must satisfy:
\begin{align}
	d_{ij}^{kl} \geq \dfrac{D^{kl}}{B_{ij}} \quad \forall (i,j,k,l): (i,j)\in\mathcal{L},(k,l)\in\mathcal{D},x_i^k=1,x_j^l=1
\end{align}

\begin{conjecture}
	\cite{lawler1973optimal} gives an optimal scheduling algorithm for a DAG on a single machine.
	With some effort, this can be proved for our case when the stage assignments have already been decided (multiple single machines with a DAG relating them all). 
\end{conjecture}

\subsection{Jointly optimization}

The Stage Assignment Problem chooses the stage assignment to minimize WAN usage cost while the Data Transfer Scheduling Problem decides whether a stage assignment can be feasibly scheduled.
Therefore, the joint optimization problem becomes finding a feasible stage assignment that minimizes the WAN usage cost.

\begin{conjecture}
	The Data Transfer Scheduling Problem can decide whether a given stage assignment has a feasible schedule or not.
	If infeasible, it also finds the sequence of bottlenecking data transfers that make it not satisfy the query deadline constraint.
	This infeasibility can be relayed back the Stage Assignment Problem's objective function with a carefully chosen dual cost.
\end{conjecture}

\subsection{Task Assignment Problem}

\subsubsection{Problem Statement}

We model each stage as a distributable group of tasks that may be fractionally assigned to multiple sites.
The fraction of tasks from stage $k$ placed at site $i$ is represented by the nonnegative decision variable $x_i^k$ and proportionally determines the size of data transfers.
This means that for any directed edge $(k,l)\in\mathcal{D}$, then $D^{kl}x_i^kx_j^l$ of data will be transfered across link $(i,j)\in\mathcal{L}$.
Note that the set of stages $\mathcal{R}$ which represent the raw data have a decision variable which is preset according to its site-wise distribution.
Each data transfer is given an upper bound on its size $d_{ij}^{kl}$ which will be used by the scheduling algorithm.

The goal of the task assignment problem is to fractionally distribute the stages of tasks to minimize the WAN usage for a given set of link duration upper bounds:

\begin{subequations}\label{eq:opt_task}
	\begin{align}
		\min_{\mathbf{x}} \quad & \sum_{(k,l)\in\mathcal{D}}\sum_{i\in\mathcal{S}}\sum_{j\in\mathcal{S}}C_{ij}D^{kl}x_i^kx_j^l \nonumber \\
		\text{s.t.}\quad & D^{kl}x_i^kx_j^l \leq d_{ij}^{kl} \quad \forall\{i,j\}\in\mathcal{S}, \forall(k,l)\in\mathcal{D} \label{eq:opt_task-duration} \\
		& \sum_{i\in\mathcal{S}}x_i^k = 1 \quad \forall k\in\mathcal{T} \label{eq:opt_task-sum1} \\
		& x_i^k \geq 0 \quad \forall i\in\mathcal{S},k\in\mathcal{T} \label{eq:opt_task-nonneg}
	\end{align}
\end{subequations}

\subsubsection{Necessary Feasibility Conditions} There needs to be a minimal amount of data flow allowed. Summing the over all pairs of sites for each $(k,l)\in\mathcal{D}$:
\begin{align}
	\sum_{i\in\mathcal{S}}\sum_{j\in\mathcal{S}}D^{kl}x_i^kx_j^l \leq \sum_{i\in\mathcal{S}}\sum_{j\in\mathcal{S}}d_{ij}^{kl} \quad\forall (k,l)\in\mathcal{D}. \nonumber
\end{align}
Since the sum of the $x$'s are 1 we have:
\begin{align}
	D^{kl} \leq \sum_{i\in\mathcal{S}}\sum_{j\in\mathcal{S}}d_{ij}^{kl} \quad\forall (k,l)\in\mathcal{D}.
\end{align}

\subsubsection{Sufficient Feasibility Conditions} \todo{}

\subsubsection{Convexifying the problem}

We are interested in convexifying the function $x_i^kx_j^l$.

When there are not linear equalities, then paper \cite{gounaris2008convexity} gives necessary and sufficient properties of functions that can be used to replace the product $x_i^kx_j^l$.
The idea is to replace $x_i^k$ with $f_1(y_1)$ and $x_j^l$ with $f_2(y_2)$ so that $F(y_1,y_2):=f_1(y_1)f_2(y_2)$ is jointly convex in $y_1$ and $y_2$.
The properties they give imply that both $f_1(y_1)$ and $f_2(y_2)$ need to be strictly convex which works well for the objective function of \eqref{eq:opt_task} and \eqref{eq:opt_task-duration} but not for \eqref{eq:opt_task-sum1} when it becomes the summation of strictly convex functions.

However when there are linear equalities that we want to keep, \cite{porn2008global} uses $x_i^k:= f_i^k(y_i^k) = e^{y_i^k}-\tau$ and then turns $x_i^kx_j^l$ into $e^{y_i^k+y_j^l}-\tau(y_i^k+y_j^l)+\tau^2$.
The caveat is that we now have the equalities $y_i^k=\ln (x_i^k+\tau)$ which are concave and not linear.
The paper uses a piecewise linear function to under-estimate the equalities.
They run it on a convex optimization solver and update the piecewise linear grid based on the previous iteration's solution.
This will numerically converge to the global optimal.

\subsubsection{A convex lower bound on $x_i^kx_j^l$} and is tight if the decision variables $x_i^k$ are binary:
\begin{align}
	x_i^kx_j^l \geq \max\{0,x_i^k+x_j^l-1\} \quad \forall x_i^k\in[0,1],x_j^l\in[0,1]
\end{align}
\begin{align}
	x_i^kx_j^l = \max\{0,x_i^k+x_j^l-1\} \quad \forall x_i^k\in\{0,1\},x_j^l\in\{0,1\}
\end{align}

\subsubsection{A convex upper bound on $x_i^kx_j^l$}
\begin{align}
	x_i^kx_j^l \leq \max\{(x_i^k)^2,(x_j^l)^2\}=(\max\{x_i^k,x_j^l\})^2 \quad \forall x_i^k\in[0,1],x_j^l\in[0,1]
\end{align}


\subsubsection{Characterizing the optima \eqref{eq:opt_task} directly}
Although Problem \eqref{eq:opt_task} is not convex, we can still find properties of the optimal solution.

By taking the Lagrange dual and the dual variables $(\boldsymbol{\lambda},\boldsymbol{\mu},\boldsymbol{\phi})$ corresponding with the respective constraints, we have the following first-order necessary stationary conditions for optimality:
\begin{align}
	\sum_{k:(k,l)\in\mathcal{D}}\sum_{i\in\mathcal{S}}C_{ij}D^{kl}x_i^k + \sum_{k:(l,k)\in\mathcal{D}}\sum_{i\in\mathcal{S}}C_{ji}D^{lk}x_i^k \nonumber \\
	+ \sum_{k:(k,l)\in\mathcal{D}}\sum_{i\in\mathcal{S}}\lambda_{ij}^{kl}D^{kl}x_i^k + \sum_{k:(l,k)\in\mathcal{D}}\sum_{i\in\mathcal{S}}\lambda_{ji}^{lk}D^{lk}x_i^k \nonumber \\
	- \mu^l - \phi_j^l = 0
	\quad \forall j\in\mathcal{S},\forall l\in\mathcal{T} \nonumber
\end{align}
Combine the 1st and 3rd summations, combine the 2nd and 4th summations:
\begin{align}
	\sum_{k:(k,l)\in\mathcal{D}}\sum_{i\in\mathcal{S}} D^{kl}(C_{ij} + \lambda_{ij}^{kl})x_i^k
	+ \sum_{k:(l,k)\in\mathcal{D}}\sum_{i\in\mathcal{S}} D^{lk}(C_{ji} + \lambda_{ji}^{lk})x_i^k \nonumber \\
	- \mu^l - \phi_j^l = 0
	\quad \forall j\in\mathcal{S},\forall l\in\mathcal{T} \nonumber
\end{align}
Relabel the index in the 2nd summation of the second summation set, and flip the summation order:
\begin{align}
	\sum_{i\in\mathcal{S}}\sum_{k:(k,l)\in\mathcal{D}} D^{kl}(C_{ij} + \lambda_{ij}^{kl})x_i^k
	+ \sum_{i\in\mathcal{S}}\sum_{m:(l,m)\in\mathcal{D}} D^{lm}(C_{ji} + \lambda_{ji}^{lm})x_i^m \nonumber \\
	= \mu^l + \phi_j^l
	\quad \forall j\in\mathcal{S},\forall l\in\mathcal{T} \label{eq:cond_stat}
\end{align}

The dual feasibility constraints give us:
\begin{subequations}\label{eq:cond_dual}
	\begin{align}
		\lambda_{ij}^{kl} \geq 0 & \quad \forall\{i,j\}\in\mathcal{S}, \forall(k,l)\in\mathcal{D} \label{eq:cond_dual_lambda} \\
		\phi_j^l \geq 0 & \quad \forall j\in\mathcal{S},\forall l\in\mathcal{T} \label{eq:cond_dual_phi}
	\end{align}
\end{subequations}

The complementary slackness constraints for the inequalities give us:
\begin{subequations}\label{eq:cond_cs}
	\begin{align}
		\lambda_{ij}^{kl}\left(D^{kl}x_i^kx_j^l - d_{ij}^{kl}\right) = 0 & \quad \forall\{i,j\}\in\mathcal{S}, \forall(k,l)\in\mathcal{D} \label{eq:cond_cs_duration} \\
		\phi_j^lx_j^l = 0 & \quad \forall j\in\mathcal{S},\forall l\in\mathcal{T} \label{eq:cond_cs_nonneg}
	\end{align}
\end{subequations}

Therefore the necessary conditions for optimality are \eqref{eq:opt_task-duration} \eqref{eq:opt_task-sum1} \eqref{eq:opt_task-nonneg} \eqref{eq:cond_stat} \eqref{eq:cond_dual} \eqref{eq:cond_cs}.

Inferences from the conditions
\begin{enumerate}
	\item At the optimal solution, the dual variables $\boldsymbol{\lambda}$ correspond with the negative gradient of the objective function w.r.t. the data transfer upper limits $\mathbf{d}$. (See \cite{bertsekas1999nonlinear} Proposition 3.3.3)
	\item At the optimal solution, $\mu^l$ is the total WAN cost plus the product of dual prices and the upper limits for all incoming and outgoing data transfers associated with stage $l$.
	Also each fraction of stage $l$ at site $j$ must share that same fraction of that stage's total cost. 
	Multiply \eqref{eq:cond_stat} by $x_j^l$ and substitute with \eqref{eq:cond_cs}:
	\begin{align}
		\sum_{i\in\mathcal{S}}\sum_{k:(k,l)\in\mathcal{D}} C_{ij}D^{kl}x_i^kx_j^l + \lambda_{ij}^{kl}d_{ij}^{kl} \nonumber \\
		+ \sum_{i\in\mathcal{S}}\sum_{m:(l,m)\in\mathcal{D}} C_{ji}D^{lm}x_j^lx_i^m+ \lambda_{ji}^{lm}d_{ji}^{lm} \nonumber \\
		= \mu^lx_j^l
		\quad \forall j\in\mathcal{S},\forall l\in\mathcal{T}
	\end{align}
	Then sum for all $j\in\mathcal{S}$ and then apply \eqref{eq:opt_task-sum1} to the RHS:
	\begin{align}
		\sum_{j\in\mathcal{S}}\sum_{i\in\mathcal{S}}\sum_{k:(k,l)\in\mathcal{D}} C_{ij}D^{kl}x_i^kx_j^l + \lambda_{ij}^{kl}d_{ij}^{kl} \nonumber \\
		+ \sum_{j\in\mathcal{S}}\sum_{i\in\mathcal{S}}\sum_{m:(l,m)\in\mathcal{D}} C_{ji}D^{lm}x_j^lx_i^m+ \lambda_{ji}^{lm}d_{ji}^{lm} \nonumber \\
		= \mu^l
		\quad \forall l\in\mathcal{T}
	\end{align}
\end{enumerate}

\subsubsection{Optimality conditions if predecessor and successor stages are pre-assigned} \todo{}

\subsubsection{Take advantage of sparcity} \todo{}