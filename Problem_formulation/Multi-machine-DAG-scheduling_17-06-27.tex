\section{Assigning and Scheduling Job Tasks on Machines}

\subsection{Model}
Suppose that there is a set of jobs $\mathcal{J}$ where each job $J\in\mathcal{J}$ can be separated into a set of individual tasks $\mathcal{T}_J$.
For a particular set tasks for a job $\mathcal{T}_J$, there is a directed acyclic graph (DAG) that represents the precedence constraints between the tasks.
A precedence constraint $j\rightarrow k$ means that task $j$ must completely finish before task $k$ can start.
Each task $i$ requires $w_i$ units of work to be done.
Let $c_i$ be time at which task $i$ is fully completed.
Therefore, the completion time for a particular job $J$ is $C_J=\max_{i\in\mathcal{T}_J}\{c_i\}$.

There is a set of machines $\mathcal{M}$ that perform work on the job tasks.
Each machine $m\in\mathcal{M}$ has a specific speed $s_m$ at which it performs work.
Therefore, if task $i$ is assigned to machine $m$, then it takes $w_i/s_m$ units of time to complete the task.
Let $W_m$ be the total amount of work that machine $m$ does and $b_m$ be the per unit cost for doing that work.

We make the following assumptions:
\begin{enumerate}
	\item Each task can only be completed by one machine but can be preempted an unlimited number of times at that machine.
	\item Each machine can only work on one task at any moment in time.
\end{enumerate}

\subsection{Problem}

The goal is to minimize the sum of the average completion time among all of the jobs and the cost of using the machines:
\begin{align}
	\min \frac{1}{|\mathcal{J}|}\sum_{J\in\mathcal{J}} C_J + \sum_{m\in\mathcal{M}} b_mW_m \label{eq:gen_prob}
\end{align}

\subsubsection{Decisions}

For each task $i$ we must assign it to a particular machine $m$ to be worked on.
Let $\mathcal{A}:\cup_{J\in\mathcal{J}}\mathcal{T}_J\rightarrow \mathcal{M}$ be the assignment.
For each machine $m\in\mathcal{M}$ we must decide a priority ordering $Q_m$ of its assigned tasks.
A machine will always be busy as long as there is an available task assigned to it that is not yet fully completed.
From a group of available tasks assigned to it, a machine will choose to work on the task that has the highest priority at any moment of time.

\subsection{Special Cases: Single machine, $|\mathcal{M}|=1$}

By restricting to $|\mathcal{M}|=1$, we eliminate the assignment decision problem.

\subsubsection{Single job, $|\mathcal{J}|=1$}

By restricting to only a single job with arbitrary precedence constraints, we only need to focus on the following minimax objective function:
\begin{align}
	\min_Q \max_{i\in\mathcal{T}}\{c_i\}
\end{align}
which can be solved in $O(|\mathcal{T}|^2)$ using a more general solution by \cite{baker1983preemptive}.
The general solution allows for arbitrary release times for each task, and the more general objective $\min \max\{f_i(c_i)\}$ where $f_i(\cdot)$ is a monotonically nondecreasing function which could include deadlines.

\subsubsection{Single-task jobs, $|\mathcal{T}_J|=1,\forall J\in\mathcal{J}$}

By restricting to jobs with only one task we eliminate the precedence constraints.
We get the following objective function:
\begin{align}
	\min_Q \frac{1}{|\mathcal{J}|}\sum_{J\in\mathcal{J}} C_J
\end{align}
which can be solved in $O(|\mathcal{J}|\log |\mathcal{J}|)$ using a more general solution by \cite{baker1974introduction} or \S 4.3.1 in \cite{brucker2007scheduling}.
The general solution allows for arbitrary release times for each job.

\subsection{If preemptions are not allowed}
Suppose we decide to restrict the first assumption to: \emph{Each task can only be completed by one machine and cannot be preempted}.  Then our problem becomes NP-hard since it is a special case in \cite{lenstra1977complexity} which was shown to be NP-complete for $|\mathcal{M}|=2$, $s_1=s_2$, $b_1=b_2=0$, $|\mathcal{J}|=1$, and no precedence constraints between the tasks for the single job.

\subsection{If you allowing tasks to be split between machines}
Suppose we decide to restrict the first assumption to: \emph{Each task can be completed by any number of machines and can be preempted an unlimited number of times at those machines.  However, a task can only be worked on by one machine at any given point of time}.  Essentially this means ``preemption" in the literature for multiple machine scheduling.

\subsubsection{Single-task jobs, $|\mathcal{T}_J|=1,\forall J\in\mathcal{J}$, and $b_m=0$ for all $m\in\mathcal{M}$}
This allows for a solution to be found in $O(|\mathcal{J}|\log |\mathcal{J}| + |\mathcal{M}||\mathcal{J}|)$ by \cite{labetoulle1984preemptive} or \S 5.1.2 in \cite{brucker2007scheduling}.

\subsubsection{Two machines and Single job, $|\mathcal{M}|=2$, $|\mathcal{J}|=1$, and $b_1=b_2=0$}
This allows for a solution to be found in $O(|\mathcal{T}_J|^2)$ by \cite{lawler1982preemptive}.

\subsubsection{Single job, $|\mathcal{J}|=1$ and $b_m=0$ for all $m\in\mathcal{M}$}
This is NP-hard even when restricted to identical machines ($s_m=s$ for all $m\in\mathcal{M}$) and all tasks the same size ($w_i=w$ for all $i\in\mathcal{T}$) \cite{ullman1976complexity}.

\subsection{Next steps}
\begin{enumerate}
	\item Combine the first two special cases so that on a single machine, we can minimize
	\begin{align}
		\min_Q \frac{1}{|\mathcal{J}|}\sum_{J\in\mathcal{J}} C_J
	\end{align}
	for multiple jobs, each with arbitrary precedence constraints.
	\item Expand the previous step to cover multiple machines starting with two machines.
	Assume that the assignment is already given.
	Find equations that describe an optimal structure.
	If the problem seems NP-hard, try to prove it.
	\item Use the equations to find to optimize for the assignment.  Maybe the problem becomes NP-hard at this point (prove it).
\end{enumerate}
